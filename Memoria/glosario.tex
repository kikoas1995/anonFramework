\chapter*{Glosario de acrónimos}
\addcontentsline{toc}{chapter}{Glosario de acrónimos}

\begin{itemize}
\item{\textbf{\textit{AES}}: Es un algoritmo de cifrado por bloques. Está entre los más utilizados y seguros actualmente disponibles. Es de acceso público.}	
\item{\textbf{\textit{Bot}}: Es un programa informático que efectúa automáticamente tareas repetitivas a través de Internet, cuya realización por parte de una persona sería significativamente más costosa.}	
\item{\textbf{\textit{Darknet}}: Subconjunto de la Deep Web que representa un espacio protegido por una red privada o al que solo un número reducido de usuario pueden tener acceso.}
\item{\textbf{\textit{Dark web}}: Conjunto de contenidos que no pueden ser indexados pues se encuentran proteigdos por sus respectivos autores, los cuales se encargan de usar y compartir dichos contenidos en redes anónimas, redes privadas, etcétera.}
\item{\textbf{\textit{Deep web}}: Contenidos que no se encuentran indexados por los principales motores de búsqueda. Son generalmente difíciles de localizar o saber de su existencia.}
\item{\textbf{Estilometría}: Consiste en el análisis de ciertos rasgos del estilo del autor para compararlo con otros textos, con el objetivo de identificar la autoría.}
\item{\textbf{GNU}: GNU's Not Unix, es un sistema operativo de tipo Unix desarrollado para el Proyecto GNU. Está formado en su totalidad por software libre.}
\item{\textbf{I2P}: Invisible Internet Project, red anónima descentralizada que permite crear servicios sobre distintos protocolos sin demasiada complejidad.}
\item{\textbf{\textit{Malware}}: Software malicioso que tiene como objetivo infiltrarse o dañar una computadora.}
\item{\textbf{NAT}: Network Address Translation, es el mecanismo utilizado los por routers para intercambiar paquetes entre dos redes que asignan mutuamente direcciones incompatibles. Consiste en convertir en tiempo real las direcciones utilizadas en los paquetes transportados.}
\item{\textbf{Ofuscación}: El término refiere a encubrir el significado de una comunicación haciéndola más confusa y complicada de interpretar.}
\item{\textbf{\textit{Pentesting}}: Pentesting o Penetration Testing es la práctica de atacar diversos entornos con la intención de descubrir fallos, vulnerabilidades u otros fallos de seguridad, para así poder prevenir ataques externos hacia esos equipos o sistemas.}
\item{\textbf{\textit{Proxy}}: Un proxy es básicamente un servidor que intermedia en las peticiones que realiza un usuario a otro servidor.}
\item{\textbf{Tor}: The Onion Router, red anónima centralizada que dota al usuario de un buen grado de anonimato.}
\item{\textbf{VPN}: Virtual Private Network, una tecnología que permite usar las ventajas de una red local sin necesidad que sus integrantes estén físicamente conectados entre sí.}
\end{itemize}

\newpage \thispagestyle{empty} % Página vacía

\addcontentsline{toc}{chapter}{Bibliografía}    %Agregamos al índice el capitulo de bibliografía 