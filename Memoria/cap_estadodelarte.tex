\chapter{Estado del arte}
\label{chap:estadodelarte}

\lsection{Privacidad y anonimato: conceptos}
\label{sec:conceptos}
La privacidad es un concepto bastante complejo, una palabra con tantas
acepciones que en algunos casos puede resultar engañosa o, incluso,
sin sentido alguno. Los temas donde se trata este
\comments{es artículo, no pronombre. No lleva tilde: verifica esto en
  el texto.} concepto van desde las leyes y derechos hasta la
tecnología, pasando por campos como la
  filosofía. \comments{Ojo, uno de los objetivos de una gran parte de
  la filosofía es evitar la ambigüedad. Quitaría este calificativo. }.

Por otro lado, el contexto en el que suele ser utilizada va desde los
ajustes de un navegador hasta uno de los debates más importantes sobre
el desarrollo de la sociedad.

En resumen, los usos del concepto de privacidad abarcan un rango
muy amplio \comments{evita hipérboles: muy amplio} de
temas y es por ello por lo que dicho término es difícil de definir. De
acuerdo a ~\cite{article:danezis2010} se tienen diversas nociones
básicas sobre la privacidad.

En primer lugar, podemos hablar de \textbf{privacidad como
  confidencialidad}, que se define como "una tradición liberal
individualista donde se le concede a la persona una autonomía libre
tanto de un estado opresor como de normas sociales". En el campo de la
informática, esto se consigue con tecnologías que abogan por la
confidencialidad de los datos, y que proporciona servicios que
minimizan, anonimizan o aseguran la colección de información personal
referente a los usuarios.

Por otro lado otra noción de privacidad, \textbf{privacidad como
  control}, la define no sólo como la ocultación de información
personal, sino también como la habilidad de controlar qué hacer con
ella~\cite{lane2014privacy}.  En este caso, la privacidad se entiende
como el derecho de un usuario a decidir qué información sobre él puede
ser difundida y bajo qué circunstancia. Este concepto de control sirve
como fundamento para los actuales sistemas
\textbf{IM}(\textit{identity management}), los cuales deben hacer un
uso razonable y comedido de la información personal de los
usuarios. Sin embargo, en un principio esto no era así.  Dichos
sistemas, los cuales nacieron en la década de 1990, no tomaban en ese
momento la privacidad como un requisito importante. Un ejemplo de ello
fue Microsoft Passport, un servicio que permitía la creación de una
cuenta de usuario, para posteriormente hacer uso de ella en múltiples
servicios de terceros. La falta de privacidad venía debido a que
Microsoft, la "proveedora de identidades", podía observar cada una de
las interacciones entre los usuarios y las aplicaciones de terceros
que usaban Passport.

Por último, podemos hablar de \textbf{privacidad como práctica}, que
se refiere a los variados mecanismos que hacen posible intervenir en
el transcurso de los datos personales, así como proporcionar de manera
transparente al usuario la forma en la que los datos son recogidos. En
este sentido, la privacidad es vista \textbf{no sólo como un derecho
  sino como un bien público}~\cite{lane2014privacy}.

En este proyecto, el concepto de privacidad nos atañe al uso
relacionado con la red, y aquí se puede definir como el control de la
información que posee un determinado usuario que se conecta a
Internet, interactuando con diversos servicios en línea con los que
intercambia datos durante la navegación.

Cabe mencionar que muchos de los usuarios que navegan día a día no son
realmente conscientes de los datos personales que circulan por la red (ver Figura~\ref{fig:estudioPew}),
y la proporción de personas que aparecen de una u otra forma en
Internet es cada día mayor~\cite{article:concernidos}.

\begin{figure}[H]
	\centerline{
		\mbox{\includegraphics[width=5.00in]{images/chart_privacy.png}}
	}
	\caption{Resultados del estudio en Pew Research Center's and American Life Project hecho en Julio de 2014~\cite{article:pew} }
	\label{fig:estudioPew}
\end{figure}

La privacidad no debe confundirse con el \textbf{anonimato en la
  red}. Este último término refiere a aquellas acciones destinadas a
garantizar que el acceso a la red se efectúa de forma que no se conoce
quien realiza la conexión. Esto es, que no sea posible establecer una
relación entre una \textbf{identidad digital} y una \textbf{identidad
  física}.

Es importante destacar que una identidad física puede estar
relacionada a una o varias identidades digitales. La relación entre
ambas identidades puede conseguirse mediante una cuenta de usuario,
así como la dirección IP desde la cual la entidad física se conecta al
servicio o incluso mediante los datos o metadatos que el ISP
(\textit{Internet Service Provider}) recoge del usuario. La extracción
de los datos por parte del ISP se puede conseguir mediante diversas
vías:

\begin{itemize}
	\item {\textbf{Monitorización:} El término refiere a la recopilación de información y la adquisición de la misma mediante diversos métodos, tales como \textit{spyware}, o intervenciones en el servicio mediante monitorización electrónico (\textit{wiretapping}).}
	\item {\textbf{Adquisición de un biproducto de otra actividad:} Entre estas actividades podemos encontrar servicios de telecomunicación, páginas y servicios web, aplicaciones, motores de búsqueda, entre otros.} 
	\item {\textbf{Uso de información ya existente}.}
\end{itemize} 

Por último conviene hablar de otra expresión que aparecerá también muy
frecuentemente en este proyecto, y es el de \textbf{ofuscación}.

En su sentido más abstracto, la  es la producción de ruido modelado en una señal existente con el objetivo de hacer una recopilación de datos más ambigua, confusa, difícil de \textit{explotar} y, por ende, menos valiosa.

\subsection{Procedimientos para incrementar el anonimato}

Este subcapítulo tiene como objetivo explicar los distintos tipos de
procedimientos para lograr un mayor grado de anonimato en la red.

Antes de nada conviene diferenciar dos conceptos comúnmente
confundidos como son el de pseudoanonimato (\textit{pseudonymity} en
inglés) y anonimato~\cite{article:anopseudo}.

El primero de ellos refiere al hecho de usar un pseudónimo con el fin
de camuflar una identidad real. Su significado literal según su
etimología es "llamado engañosamente".  Por otra parte, el segundo,
cuyo origen etimológico significa "sin nombre", refiere cuando no hay
información identificable a nada ni nadie.

Por ende, la principal diferencia entre ambos términos radica en que
mientras en el anonimato la identidad es totalmente desconocida, en el
pseudoanonimato se aprovecha el hecho de utilizar un pseudónimo para
esconder una identidad real.  En este proyecto se ha trabajado con
métodos tanto para lograr un anonimato como para lograr un
pseudoanonimato. \comments{En este respecto conviene  retomar la
  cuestión relativa a la asociación entre identidad física e identidad
digital. Dichio vínculo se establece mediante tres tipos de
mecanismos: algo que el usuario conoce, algo que el usuario que posee
y algo que el usuario es. En este trabajo se analizan procedimientos
para disociar la identidad física de procedimientos de autenticación
basados en localización mediante la dirección IP de un usuario, así
como de mecanismos de validación de usuarios mediante contraseña. }

Una vez aclarados sendos términos, procedemos a listar las diferentes vías para lograr el anonimato:

\begin{itemize}
	\item Anonimato de emisor: Este tipo consiste en un origen que efectúa un mensaje a un determinado receptor, y el emisor no puede ser reconocido por ningún observador.
	\item Anonimato de receptor: En este caso, al contrario del anterior, es el receptor el que no puede ser identificado por el observador.
	\item Anonimato de comunicación: Esto involucra una no vinculación entre emisor y receptor, de forma que es posible identificarlos a ambos por separado pero no la asociación entre ellos.

\end{itemize}

\lsection{Hechos relevantes con respecto a la seguridad en la red} \label{sec:historia}

Conviene empezar el capítulo recalcando que Internet no fue concebido como un protocolo de comunicación seguro. 
Es por esto que a lo largo de su historia han ocurrido varios sucesos que han puesto en riesgo (por diversos motivos) la seguridad del usuario en la red.

la privacidad del usuario se ha visto comprometida en numerosas ocasiones debido a los robos de información personal, en la mayor parte de ocasiones debido al \textit{malware}. Es por ello que esta sección estará dedicada a dar un repaso exhaustivo a la historia de la seguridad en la red para comprender cuál es el foco de los distintos ataques actualmente.

Podemos marcar como primera incursión histórica con respecto a la seguridad un libro publicado por Jon Von Neumann en el año 1949 llamado \textit{The Theory of Self Reproducing Automata}. Dicha publicación sirvió como  base para el desarrollo de los primeros \textbf{virus informáticos}~\cite{article:automata}.

El primer virus que causó un gran impacto social fue el lamado \textbf{virus Creeper}, el cual era un programa experimental autoreplicante creado por \textbf{Bob Thomas} diseñado con fines experimentales y que no causaba un daño real entre las máquinas en las que se iba moviendo ~\cite{article:motivationvirus}.
Realmente fue desplazándose de ordenador en ordenador alrededor de toda la red ARPANET (la precursora de lo que es a día de hoy internet).

En 1973 Robert Metcalfe, un trabajador de ARPANET y el cuál fundó 3Com (uno de los fabricantes de redes informáticas más importantes), advertía que una incursión a la red interna desde el exterior era algo extremadamente sencillo y, de hecho, son atribuidas durante la década de los 70 varias intrusiones a la red por parte de estudiantes de secundaria. 
Durante esta etapa no se produjeron descubrimientos destacables con respecto a la seguridad informática. De hecho, en el año 1978 un grupo de científicos propusieron un proyecto de cifrado de paquetes TCP/IP pero encontraron muchas trabas, algunas de ellas incluso por la Agencia de Seguridad Nacional~\cite{article:nsa_tcp}. Por ello, dicho proyecto (que bien podría haber marcado otro camino en la historia de la seguridad en la informática) fue abandonado. 

En 1981 apareció el segundo virus reconocido a nivel mundial, el llamado \textbf{Elk Cloner}~\cite{article:cloner}. Atacaba computadoras Apple II, aunque su único propósito era el de reproducirse en otros dispositivos y no efectuaba ningún daño propiamente dicho. Uno de los datos más impactantes es que fue diseñado por un joven de 15 años. Su propagación era mediante el disquete. 
Este virus sentó la base para los siguientes que fueron apareciendo, los cuales contendrían todo tipo de código destructivo (robo de información, manipulación de los mismos, destrucción de software y hardware...) y se propagarían por más medios, como el correo electrónico e Internet.
Debido a la aparición de los numerosos softwares maliciosos en esta época fueron apareciendo empresas que proporcionaban herramientas para proteger los equipos, los \textbf{antivirus}.

En el año 1983, se hizo obligatorio que los usuarios de la red ARPANET utilizasen el protocolo TCP/IP. Este hecho estableció un estándar en la comunicación entre redes y favoreció la aparición de la World Wide Web.
Este fue además el año en el que se utilizó por primera vez el término \textit{virus informático} en una tesis académica, dirigida por Fred Cohen~\cite{article:cohen}.

En 1986 se aprueba una ley llamada La \textit{Ley De Fraude Y Abuso Cibernético}~\cite{article:ley} la cual aparece como contramedida al virus más dañino hasta la fecha como fue Brain~\cite{article:brain}, el primer virus compatible con máquinas IBM. La ley defendía a los usuarios del \textbf{robo de datos}, del acceso a la red no autorizado y demás delitos relacionados con la tecnología.

Los tres años siguientes siguieron apareciendo compañías que velaban por la seguridad como Symantec, la cual lanzó el conocido antivirus Norton en el año 1991.
Sin embargo, esto no hizo que el número de robos de información cesara. Al contrario, pues con la aparición del primer navegador web surgieron nuevas formas de ataque y \textit{phising}. Asimismo surgieron los primeros ataques de denegación de servicio.

En el año 1996, con la aparición del complemento de navegador Flash(que permite la reproducción de vídeo y música) surgen por ende nuevas formas de ataques, debido a las numerosas vulnerabilidades del plug-in. El correo electrónico, que cada vez se encuentra más vigente, también permite recibir correspondencia con un objetivo de robar información personal. En este año también aparece el primer virus creado para el sistema Linux~\cite{article:staog}, llamado Staog.

El año 2000 el número de gusanos informáticos que se encuentran en equipos domésticos es desproporcionado. Asimismo, en esta década los ciberdelincuentes aprenden a anonimizarse más sofisticadamente y hacen más difícil su identificación.
Para el año 2005 el número de malware únicos asciende a más de 300.000, y ha ascendido más de un 1000\% en diez años. Para el año 2008 dicha cifra asciende 5 millones de malware únicos~\cite{article:number_malware}.

En los últimos años la variedad de ataques de robo de información ha cambiado mucho. Han habido noticias recientes con elementos tecnológicos tan cotidianos como la empresa fabricadora de robots-aspiradora Roomba, la cual vende información a terceros sobre los planos de tu casa~\cite{article:roomba}. 

Con la aparición de los smartphones, surgen nuevas amenazas para nuestra seguridad y privacidad como \textit{payloads} que corren como un proceso en segundo plano~\cite{article:android} y permiten tracear nuestra ubicación, espiarnos a través de la cámara del dispositivo y un largo etcétera. Esto ha ocasionado que surjan también aplicaciones antivirus en nuestros teléfonos móviles, que actúan de forma \textit{pasiva} vigilando que ningún proceso monitorice de forma maliciosa nuestro dispositivo. 

Como se ha mencionado, está claro que debemos cuidarnos de agentes externos como los distintos tipos de \textit{malware}, los cuales pueden usurpar nuestra identidad, comprometer nuestros \textbf{datos personales}, entre otras cosas. 
Sin embargo, \textbf{no todo depende del usuario }puesto que muchos de los servicios que utilizamos día a día han sufrido de \textit{leaks} de bases de datos e infiltraciones en el sistema que suponen un riesgo para nuestra privacidad. Un servicio muy útil para comprobar si hemos sido víctimas de ciberataques (con el riesgo que supone para nuestra privacidad) a sitios donde frecuentamos es \url{https://haveibeenpwned.com/}.

Conforme ha ido avanzando la tecnología, el número de \textit{malware} ha ido creciendo exponencialmente, y así también su propósito. Mientras que en un principio la gran mayoría eran pruebas de concepto, en los últimos años el objetivo ha cambiado su foco hacia la información sensible de los usuarios de la red. 

Asimismo, hemos podido comprobar que no sólo el \textit{malware} es causante de los robos de información, sino que la propia negligencia de algunas empresas en temas relativos a la privacidad de los usuarios también juega un papel importante y es por ello que no debemos perder de vista que, aparte de proteger la confidencialidad de la información, también se debe hacer lo propio con la privacidad.

Por todo esto, con el incremento del número de amenazas a nuestra privacidad también ha sido necesario que aparezcan herramientas que actúen de forma \textit{activa} y nos permitan utilizar los servicios de Internet de una forma más segura.
	
\lsection{Herramientas que ofuscan la identificación del usuario}
\label{sec:a_favor}

A continuación se menciona varias tecnologías utilizables a la fecha de redacción de este documento para ayudar a un usuario a proteger su identidad en Internet y anonimizarse mediante \textbf{ofuscación del tráfico de red}, utilizando \textit{proxies}, VPNs, redes anónimas o incluso dificultar la identificación inequívoca del mismo haciendo uso de \textbf{herramientas basadas en otros tipos de ofuscación}, como la de estilometría.

Hoy en día existen numerosas aplicaciones que permiten aumentar nuestro grado de anonimato a la hora de realizar tareas en Internet, usualmente a costa de una más óptima velocidad de conexión,por ejemplo, el servicio \textbf{Tor}, o a cambio de una suscripción temporal, como los servicios VPN de pago.

Cada una de estas medidas tiene sus puntos fuertes y sus contrapartidas, y serán tratadas en este subcapítulo.

\subsection{Servidor proxy}

Un servidor proxy es básicamente un mediador entre un usuario que realiza una petición y otro servidor.
Su funcionamiento~\cite{article:proxy} es relativamente simple: Cuando un cliente de la red desea acceder a un recurso, es el servidor proxy el que realiza la comunicación y el que lleva el resultado de la petición al usuario final. 

Los usos de dicha aplicación en ejecución van desde aumentar el rendimiento de algunas operaciones sirviendo como memoria caché, hasta proteger la identidad del usuario que lo utiliza.
Dicha finalidad depende del tipo de proxy que se esté utilizando.
Hay varios tipos, los cuales se resumen a continuación.

\subsubsection{Proxy Web}

Un \textbf{servicio proxy o proxy web} es un proxy para una aplicación concreta, y permite el uso de los protocolos FTP y HTTP/S.

Este tipo de proxy es muy comúnmente utilizado para proteger la privacidad y se puede utilizar junto con Tor (el cual veremos más adelante) para mejorar el grado de anonimato en la red.

El esquema de funcionamiento es el siguiente (ver Figura~\ref{fig:web_proxy}):

\begin{figure}[H]
	\centerline{
		\mbox{\includegraphics[width=3.00in]{images/proxy_web.png}}
	}
	\caption{Funcionamiento de un Web proxy o Proxy service}
	\label{fig:web_proxy}
\end{figure}

\subsubsection{Proxy Cache}

Su propósito es el de guardar el contenido solicitado por el usuario para así mejorar la velocidad de respuesta en futuras solicitudes de recursos. 
Conviene destacar a su vez que un proxy web puede actuar también almacenando las páginas web solicitadas, actuando de cierta manera como un proxy caché. Funciona de la siguiente manera (ver Figura~\ref{fig:proxy_cache}):

\begin{figure}[H]
	\centerline{
		\mbox{\includegraphics[width=3.00in]{images/proxy_cache.png}}
	}
	\caption{Funcionamiento de un Proxy Caché}
	\label{fig:proxy_cache}
\end{figure}

\subsubsection{Transparent Proxy}

También es conocido como proxy forzado, y tiene la peculiaridad de no modificar la petición realizada por el cliente o respuesta más allá de la autenticación del propio proxy.
Se le llama transparente puesto que el usuario final no necesita realizar ningún tipo de configuración adicional en el navegador.
Su uso es principalmente el de filtrar ciertas conexiones (se combina con un \textit{cortafuegos}) y para proporcionar seguridad (ver Figura~\ref{fig:trans_proxy}).

\begin{figure}[H]
	\centerline{
		\mbox{\includegraphics[width=3.00in]{images/proxy_transparent.png}}
	}
	\caption{Funcionamiento de un Proxy transparente~\cite{article:proxy_trans}}
	\label{fig:trans_proxy}
\end{figure}

\subsubsection{Reverse Proxy}

Este proxy tiene la peculiaridad de estar alojado en uno o más servidores web.
Es decir, mientras que un proxy normal es el intermediario entre sus clientes para realizar peticiones a cualquier servidor, un proxy inverso es el intermediario entre sus servidores asociados para ser contactados por cualquier cliente~\cite{article:proxy_rev}.

\subsubsection{NAT proxy o enmascaramiento}

El uso de este proxy es también llamado \textbf{enmascaramiento de IP}. En este caso, las direcciones de destino de los paquetes IP son reemplazadas por otras~\cite{article:nat}.
En este caso la actuación de mediador es entre los equipos de la red interna y la red exterior.\\

Los aquí citados son los principales tipos de proxies. No obstante, existen algunos más, como el \textbf{proxy abierto} y el \textbf{Cross-Domain }proxy.
Una vez dejados claros los conceptos básicos sobre proxies, mencionaremos algunas herramientas útiles que hacen uso de estos para navegar de una forma más anónima y segura por la red.

\subsubsection{Proxychains}

Proxychains es un programa disponible únicamente para GNU/Linux y Unix que nos permite crear cadenas de proxies, escondiendo así nuestra dirección IP pública en \textbf{todo tipo de conexiones} (HTTP, FTP, SSH, etcétera). 
Esto se traduce en que podemos navegar por Internet o realizar cualquier operación en la red de redes sin descubrir nuestra identidad real.

Mientras que en las figuras anteriores mostramos una conexión a la red con la utilización de un sólo proxy, en el caso de \textit{proxychains} (como su propio nombre indica) utilizaremos cadenas de servidores proxy (ver Figura~\ref{fig:proxy_chain}) para anonimizar~\cite{article:proxychains} el tráfico que generemos(no necesariamente debe ser tráfico web).

\begin{figure}[H]
	\centerline{
		\mbox{\includegraphics[width=3.00in]{images/proxy_chain.png}}
	}
	\caption{Ejemplo de una cadena de proxies}
	\label{fig:proxy_chain}
\end{figure}

Para hacer funcionar proxychains en un equipo es necesario modificar su fichero de configuración (\textit{proxychains.conf}). En él existen varias opciones en cuanto a la formación de ccadenas de proxies:

\begin{itemize}
	\item \textit{Dynamic chains}: Supongamos que tenemos 4 servidores proxies añadidos en nuestro archivo de configuración, en este orden: A, B, C y D.
	En el caso de que, por ejemplo, el servidor B esté caído y el resto funcionen perfectamente, la conexión se realizará de la siguiente manera (ver Figura~\ref{fig:dyn_chain}):
	
	\begin{figure}[H]
		\centerline{
			\mbox{\includegraphics[width=3.00in]{images/proxy_dynamic.png}}
		}
		\caption{Cadena dinámica}
		\label{fig:dyn_chain}
	\end{figure}
		
	Es decir, aunque uno (o varios) de los servidores proxy que componen la cadena no funcione, siempre se intentará realizar la conexión omitiéndolo.
	
	\item \textit{Strict chains}: Si tomamos el ejemplo anterior, en el caso de una cadena estricta ocurre lo siguiente (ver Figura~\ref{fig:strict_chain}):
	
	\begin{figure}[H]
		\centerline{
			\mbox{\includegraphics[width=1.80in]{images/proxy_strict.png}}
		}
		\caption{Cadena estricta}
		\label{fig:strict_chain}
	\end{figure}
			
	En el caso de que uno de los servidores proxy falle, la conexión no será satisfactoria. Por ello, las cadenas estrictas tienen la peculiaridad de que siguen el orden de los servidores rigurosamente.
	
	\item \textit{Random chains}: Este tipo de cadenas es totalmente distinta de las anteriores. Básicamente escoge uno de los servidores proxy que aparecen en el archivo de configuración de forma aleatoria.
\end{itemize}

Ahora bien, ¿cómo podemos añadir servidores proxy a nuestra cadena?
El formato para añadirlos es el siguiente:


{\fontfamily{bch}\selectfont 
	socks5  192.168.67.78   1080    user   password
}

El primer elemento es el tipo de proxy. Las posibilidades son HTTP, socks4 y socks5. Como nuestro objetivo es el de anonimizarnos y proteger nuestra identidad \textbf{siempre que sea posible se intentará utilizar socks5}.
El segundo y tercer elemento es la dirección IP y el puerto del proxy, respectivamente. 
Por último, el cuarto y quinto campo son opcionales y depende de si el proxy que utilizaremos cuenta con usuario y contraseña. Normalmente cuentan con contraseña los proxies que adquirimos por medio de plataformas de pago.

Hay una gran cantidad de páginas que ofrecen servidores proxies, tanto gratuitos como de pago. La diferencia radica en la carga de dichos servidores. Muchos de los servidores proxy gratuitos publicados en la red se encuentran saturados y limitan mucho la velocidad de conexión.

No obstante, es posible encontrar servidores proxy gratuitos que funcionan relativamente bien. Una buena página en la que encontrarlos es \url{https://socks-proxy.net/}. 
Esta página permite visualizar servidores proxy gratuitos con disponibilidad actualizada cada 20 minutos. Además, permite filtrarlos según tipo y sobre todo, país de origen.

Con el objetivo de mantenernos anónimos en la red, conviene utilizar proxies ubicados en países que tengan buenas políticas de privacidad. Ejemplo de ello son países como \textbf{Rusia, China o Países Bajos}. 

Otros países, como Estados Unidos, Reino unido, Nueva Zelanda, Australia o Canadá \label{key}(grupo conocido como los Cinco Ojos~\cite{article:politicas}), están enfocados en recopilar y analizar datos entre sí, por lo que \textbf{no conviene utilizar} servicios basados en estas ubicaciones si el objetivo es mantener nuestra privacidad a salvo.  

\subsection{VPN}

Una VPN (o red virtual privada) no es más que el uso de una red privada segura sobre una red pública más grande. 
La notoriedad que ha cosechado estos últimos años~\cite{article:notoriedad} reside en que nos permite gozar de un alto grado de anonimato al darnos acceso al envío y recibo de datos de la red pública, teniendo todas las políticas de privacidad de una red privada.

La forma de conseguir esto es, normalmente, estableciendo una conexión extremo a extremo mediante el uso de cifrado y/o conexiones dedicadas.

Pese a que el término se ha visto utilizado enormemente estos últimos años, lo cierto es que las redes privadas virtuales existen desde hace bastante tiempo.
De hecho, la primera forma de VPN surgió con SwIPe (\textit{Software IP Encryption Protocol})~\cite{article:swipe}, un trabajo experimental surgido en el año 1993 por John Ioannidis y su equipo en la Universidad de Columbia y AT\&T Labs. Este proyecto pretendía garantizar confidencialidad, integridad y autenticación del tráfico de red.

Tras este experimento, en el año siguiente Xu Wei continuó investigando acerca de la seguridad del protocolo IP hasta formar la familia de protocolos IPSec~\cite{article:ipsec}, la cual autentica y cifra cada paquete compartido a través de una red pública. Después de un tiempo y tras la mejora en las velocidades de transmisión de paquetes, y de la función \textit{plug-and-play}, fue posible la salida al mercado de \textbf{las primeras VPNs}.

A la vez que apareció IPSec, se  realizó un trabajo en la Biblioteca de Investigación NAVAL con ayuda de DARPA (Defense Advanced Research Projects Agency) con el que surgió el \textbf{Protocolo de Seguridad de Encapsulación}~\cite{article:levpn}. 
Con ello surgió un gran avance para la seguridad en internet y la tecnología VPN. 
La Carga de Seguridad Encapsulada~\cite{article:esp}, \textbf{ESP, ofrece la autenticidad, integridad y protección de la confidencialidad de los paquetes de datos}. 
Permite configuraciones de autenticación, encriptación, o ambos. 
Este protocolo es similar al de los Encabezados de Autenticación y proporciona una segunda capa de seguridad para las conexiones a Internet. 

1995 fue el año en el cuál se creó el grupo de trabajo de IPsec dentro de la IETF~\cite{article:ietf}, o Internet Engineering Task Force, el cual es una comunidad de ingenieros de Internet, proveedores, desarrolladores y otras personas interesadas en la evolución de internet y su buen funcionamiento. 

El objetivo de este grupo era el de crear un conjunto estandarizado de protocolos disponibles libremente y examinados abordando los componentes, extensiones y la implementación de IPsec.

IPsec está formado por tres subprotocolos:

\begin{itemize}
	\item \textit{Autentication Header (AH)}: Este protocolo es el encargado de proporcionar integridad de datos en el caso de no haber conexión y autenticación de paquetes IP, además de protección contra ciertos tipos de ataques. 
	La \textbf{autenticación} es importante porque asegura que los paquetes de datos que envías y recibes son los que deseas, no el malware u otros ataques potencialmente dañinos. Hay varias versiones con diferentes grados de protección a diferentes niveles. En todos los casos, de esta manera tus datos personales están protegidos.
		
 	\item \textit{Encapsulating Security Payload (ESP)}: Se ocupa de proporcionar la confidencialidad de esos paquetes, al igual que integridad de origen de los datos, la seguridad a los ataques y también seguridad para el tráfico de flujo. Cuando se utiliza en \textbf{Modo Túnel}, proporciona seguridad para todo el Paquete IP.
 	
	\item \textit{Security Associations (SA)}: Son algoritmos y datos que permiten que AH y ESP funcionen correctamente. 
	Básicamente, los datos se cifran en paquetes en la fuente y luego se transfieren a través de internet de forma \textbf{anónima} para ser recibidos, autenticados y descifrados en el destino. Las asociaciones se crean sobre la base de la Internet Security Association And Key Management Program (ISKAMP) utilizando una serie de números.
\end{itemize}

Además de esto, existen dos modos de funcionamiento~\cite{article:ibm_ip}:

\begin{itemize}
	\item \textit{Transport Mode}: En Modo Transporte únicamente la carga Útil de IP es típicamente cifrada asegurando los datos, pero dejando visible la información que se origina.
	
	\item \textit{Tunneling Mode}: En Modo de Túnel todo el Paquete IP está cifrado y encapsulado, se le otorga un nuevo encabezado de autenticación y luego se envía. Modo túnel es la tecnología que impulsa la VPN de hoy en día.
\end{itemize}

Veamos una comparativa entre un paquete IP original, uno que hace uso del modo transporte y otro del modo túnel de IPsec (ver Figura~\ref{fig:IPtypes}):

	\begin{figure}[H]
		\centerline{
			\mbox{\includegraphics[width=5.00in]{images/IPsec.png}}
		}
		\caption{Ejemplo de paquetes IP con diferentes niveles de seguridad de encapsulación~\cite{article:ibm_ip}}
		\label{fig:IPtypes}
	\end{figure}

El protocolo de túnel hace de las VPNs una gran opción si el objetivo es la protección de nuestra información personal. 
Permite al usuario conectarse a Internet con una dirección IP que no es parte de la red local. La \textbf{tunelización} funciona encriptando y encapsulando los datos, es decir, lo que proporciona un tercer y muy buscado beneficio: \textbf{el anonimato y la privacidad}. 

La forma en la que opera es un poco más compleja que con el modo de transporte, los paquetes que contienen la información que realiza el cifrado y el servicio de entrega se llevan a cabo dentro de la carga útil del mensaje original, pero operan a un nivel más alto que la propia carga útil, creando un escudo formado desde dentro y seguro de las influencias externas. Los mejores servicios cifrarán todo el paquete, el marcador de identificación y todo; a continuación, volverán a encapsularlo con una nueva dirección IP y marca de identificación para obtener una completa privacidad.

El modo túnel recoge su nombre de la forma en la que una red VPN opera (ver Figura~\ref{fig:tunnelMode}):

  		\begin{figure}[H]
  			\centerline{
  				\mbox{\includegraphics[width=3.50in]{images/vpn.png}}
  			}
  			\caption{Esquema del \textit{Tunneling mode}~\cite{article:tunnelMode}}
  			\label{fig:tunnelMode}
  		\end{figure}
 
 Hoy en día, se usan diferentes tecnologías VPN, cada una con sus pros y sus contras. Entre todas ellas las más destacables son:
 
 \begin{itemize}
 	\item PPTP: \textit{Point to Point Tunneling Protocol}, la cual está bajo licencia de Microsoft(fue el primer protocolo de VPN compatible con Windows), crea una red privada virtual en redes dial-up. ~\cite{article:pptp}
 	
 	Su implementación requiere poca sobrecarga de cómputos, lo cual lo hace uno de los protocolos de VPN \textbf{más rápidos} disponibles actualmente.
 	El problema con esta tecnología reside en que \textbf{no es del todo segura}. Aunque ahora normalmente utiliza una encriptación de 128 bits, existen varias vulnerabilidades de seguridad, con la posibilidad de una autenticación MS-CHAP v2 no encapsulada como la más grave. Con todo esto, una red que usase PPTP podría ser decodificada en apenas días.
 	La misma Microsoft, pese a haber corregido el fallo de seguridad, no recomienda el uso de este protocolo, y recomienda el uso de SSTP o L2TP.
 	
 	\item L2TP y L2TP/IPsec: El protocolo de túnel de capa dos ~\cite{article:pptp}, normalmente se implementa con los protocolos IPsec (explicados anteriormente) para encriptar datos antes de la transmisión, a fin de proveer a los usuarios privacidad y seguridad. Todos los dispositivos y sistemas operativos modernos compatibles con VPN tienen L2TP/IPsec incorporado. La configuración es tan rápida y fácil como la de PPTP, sin embargo en ocasiones puede ser problemático en el caso de usar un cortafuegos NAT restrictivo.
 	Por el momento, \textbf{no hay vulnerabilidades importantes }relacionadas con la encriptación por IPsec, pero John Gilmore,  miembro fundador y especialista en seguridad de la Electric Frontier foundation, afirma que es probable que el protocolo sea debilitado intencionalmente por la NSA.
 	
 	\item SSTP: Secure Socket Tunneling Protocol~\cite{article:sstp} fue presentado por Microsoft en el Service Pack 1 de Windows Vista. Este estándar está además ahora disponible para SEIL, Linux y RouterOS, aunque sigue siendo principalmente una plataforma únicamente para Windows. 
 	Utiliza SSL v3, y no tendría por qué tener problemas de seguridad aparentes. Sin embargo, hay que recordar que es propiedad de una empresa gigante como Microsoft, y no puede ser analizado en busca de ingresos clandestinos.
 	
 	\item IKEv2: Internet Key Exchange~\cite{article:sstp}, en su segunda versión, es un protocolo de túnel basado en IPsec, fue desarrollado por Cisco y Microsoft.
 	Los dispositivos móviles son los más beneficiados con IKEv2 ya que el protocolo de movilidad y multioproveedor que se ofrece en forma predeterminada lo hace extremadamente flexible para cambiar de redes. Pese a que IKEv2 está disponible en menos plataformas comparado con IPsec, tiene buena reputación en términos de estabilidad, seguridad y rendimiento.
 	
 	\item OpenVPN: Es un estándar \textit{open-source }relativamente nueva, utiliza los protocolos SSLv3/ TLSv1 y biblioteca OpenSSL para brindar a los usuarios una solución de VPN confiable y potente~\cite{article:sstp}. El protocolo tiene amplia capacidad de configuración, lo que hace que sea muy difícil de bloquear para servicios como Google.	La principal ventaja de esta tecnología es que OpenSSL, la biblioteca que utiliza, soporta\textbf{ múltiples algoritmos criptográficos }tales como 3DES, AES, Camellia, Blowfish, CAST-128 y más, aunque Blowfish o AES son utilizados casi exclusivamente por proveedores de VPN.
 	La rapidez con la que se desempeña el protocolo OpenVPN depende del nivel de encriptación utilizado, pero normalmente es más rápido que IPsec. Por contra, la configuración es complicada en comparación con L2TP/IPsec y PPTP.
 \end{itemize}
 
 \subsubsection{OpenVPN}
 
	Una vez explicadas las diferencias de este estándar con algunas de sus alternativas, vamos a ver cómo funciona este en una plataforma GNU/Linux.
	
	Lo primero que conviene hacer es cambiar \textbf{el servidor DNS por defecto.}
	Esto, pese a que no es algo explícito ni directamente relacionado con el funcionamiento de la VPN, sí es recomendado. En determinadas ocasiones, pese a utilizar un servicio VPN, la traducción de nombre de dominios a su correspondiente IP numérica (dicha petición debería hacerse mediante el túnel VPN) puede hacerse erróneamente por medio del proveedor de Internet. Esto se conoce como \textbf{DNS leak}. 
	Para evitarlo hay muchas soluciones. Entre ellas, algunos clientes de VPN (como Mullvad) permiten activar un campo en la configuración que evita estos problemas. Otra forma, en sistemas Unix, es acceder al archivo de configuración localizado en /etc/dhcp/dhclient.conf, descomentar la línea:
	
	{\fontfamily{bch}\selectfont 
		\#prepend domain-name-servers 127.0.0.1;
	}

	Evidentemente, hay que cambiar la dirección del servidor de DNS que utilizaremos. Podemos encontrar múltiples servidores, todos ellos seguros, en \url{https://www.opendns.com}.
	
	Tras esto (y reiniciar la red, evidentemente) podemos utilizar openVPN sin riesgo a que ocurran DNS-leaks.
	
	Utilizar openVPN es tan sencillo como llamar al programa por línea de comandos pasándole como argumento un fichero \textit{.ovpn}, los cuales podemos conseguir, por ejemplo, en \textit{openvpn.com}.
	
	Una duda que puede surgirnos es, si nuestro objetivo es mejorar nuestra privacidad y anonimato ¿cuál es una mejor alternativa, una VPN o el uso de un proxy? 
	
	Lo cierto es que, como veremos, no son las únicas opciones a la hora de obtener un mayor grado de anonimización. Sin embargo, la respuesta no es rotunda puesto que depende de qué servicio VPN y qué proxy se utilice.
	
	En el uso de un proxy, hay tres protocolos principales, que como hemos visto son HTTP/HTTPS y SOCKS.
	Si usamos un proxy HTTP o SOCKS, el uso de este no proveerá de ningún tipo de encriptación de los datos, mientras que los proxies HTTPS ofrecen un nivel de encriptación igual que una web que funcione con el protocolo SSL. Además, por lo general, el uso de los proxies (sobre todo si se usan cadenas de los mismos) implican una muy baja velocidad de conexión.
	
	El caso de los VPN, hasta el momento es imposible interceptar el tráfico que circula por su túnel. Sin embargo, esto produce un \textbf{único punto de fallo}, y es el servicio VPN. Es necesario asegurarse de que el servicio VPN que utilizamos no guarda logs ni otros datos, pues en el momento en el que dichos logs saliesen a la luz, estaríamos totalmente expuestos.
	
	Existen multitud de servicios VPN (cada uno con sus pros y sus contras) que pueden utilizarse mediante OpenVPN. Los servicios más relevantes que defienden la privacidad del usuario por encima de todo, a fecha de redacción de este documento son:
	
	\begin{itemize}
		\item {\textbf{ExpressVPN}}: Servicio de pago que proporciona una conexión mediante VPN y asegura \textbf{no registrar logs de actividad ni de conexión}, lo que garantiza una mayor privacidad para el usuario. Contiene VPN  Sus VPN están ubicadas en 78 países distintos.
		\item {\textbf{BlackVPN}}: BlackVPN tampoco guarda logs de actividad ni de conexión. Además el servicio está basado en Kowloon y actúa bajo la jurisdicción de Hong Kong, lugar donde se cuida el derecho a la privacidad del usuario.
		\item {\textbf{TorGuard}}: Este servicio de nuevo no registra ningún tipo de logs. Además, utilizan una configuración de IP compartida entre todos sus servidores, lo que en conjunto hace imposible establecer una relación entre una persona física y una dirección IP en una fecha concreta. Cuenta con miles de servidores en 49 países distintos.
		
	\end{itemize}
	
	También existen servicios de VPN más sencillos de configurar que actúan mediante extensiones de navegador:
	
	\begin{itemize}
		\item {\textbf{Hoxx}}: Plug-in disponible para Chrome y Firefox (aunque también cuenta con una aplicación Android oficial) totalmente gratuito. Promete ser un servicio sin límite de banda ancha, que permite anonimizar tu conexión e impedir que el tráfico sea interceptado.
		\item {\textbf{Tunello}}: Al igual que Hoxx, Tunello es un plug-in de navegador (disponible en Firefox y Chrome) que proporciona un servicio VPN. Tiene un límite de uso de 6 GB de datos mensuales gratuitos y tampoco tiene límite de velocidad.	
	\end{itemize}

\subsection {Deep-web}
Pese a no ser algo novedoso, el término de la web profunda se ha ido popularizando y extendiendo tanto en la comunidad \textit{hacker} como entre usuarios comunes en Internet 	y, a menudo, es usado de manera incorrecta ya que se suele confundir con la \textit{dark-web}. 

Para empezar, la \textit{deep-web}~\cite{article:deepweb} se refiere a aquellos contenidos que no están indexados por los prinicipales motores de búsqueda, tales como Google o Bing. Por ende, es bastante complicado localizarlos, ni tan siquiera saber de su existencia.
Los motivos de no encontrarse indexados son muy variados. Suele ser debido a que el contenido se encuentra protegido por una contraseña, se encuentra en una VPN a la cual no tienen acceso los \textit{crawlers} lanzados por los buscadores, o simplemente son tan antiguos e irrelevantes que no aparecen en las consultas de los buscadores.
Cabe mencionar que algunos de los contenidos sí se encuentren indexados, pero dadas sus características, no aparezcan con los criterios de búsqueda convencionales utilizados por los usuarios.

Por otro lado, la web oscura (o \textit{dark-web}) refiere a contenidos que no es posible indexar debido a que se encuentran protegidos por sus autores, los cuales se encargan de compartirlos en redes anónimas o sitios web protegidos con contraseña. 
No existe una finalidad específica para estos contenidos, pero normalmente se trata de páginas web encargadas para la administración de un portal o contenido relacionado con actividades ilegales.

Las dimensiones que abarcan los contenidos de la web profunda son enormes, y actualmente no hay forma de medir (ni tan siquiera de forma aproximada)	dicha cantidad de información.
Una imagen sumamente extendida en Internet representa la relación entre los contenidos de la web profunda y la parte visible de Internet, afirmando que la \textit{deep-web} constituye el 96\% de Internet. De nuevo, esto no es cierto, puesto que no hay manera de medir, ni tan siquiera con un margen de error aceptable, la cantidad de información disponible en la web profunda. Muchos contenidos de esta o bien son desconocidos, o bien no se encuentran disponibles durante largos períodos de tiempo.

Otro término muy utilizado es el de \textit{darknet}. Una \textit{darknet }es un subconjunto de la web profunda que representa un espacio protegido por una VPN, o bien al que sólo un número reducido de usuarios autorizados pueden acceder. Los contenidos no son indexados por ningún buscador. Es más, en ciertos casos las direcciones de los servicios \textbf{no son resolubles por medio de mecanismos tan habituales como consultas DNS.} Para poder navegar en estas redes normalmente es necesario hacer uso de un cliente, como bien puede ser el de Tor.

En numerosas ocasiones se relaciona dicho tipo de redes con la ciberdelincuencia. Esto es debido a que estas redes proporcionan un muy buen grado de privacidad y anonimato al usuario, lo cual es aprovechado para realizar actividades ilegales tales como la venta de armas o drogas. 
De hecho, un estudio realizado entre Diciembre de 2013 y Julio de 2015 por un investigador apodado Gwern Branwen sobre las ventas en numerosos mercados de las \textit{darknets}~\cite{article:gwern} revela que se llegaron a recaudar más de 27 millones de dólares en drogas ilegales (ver Figura~\ref{fig:gwern}).

\begin{figure}[H]
	\centerline{
		\mbox{\includegraphics[width=2.70in]{images/darknet_markets.png}}
	}
	\caption{Imagen extraída del artículo \textit{The data of the dark web - The Econommist}~\cite{article:gwern}}.
	\label{fig:gwern}
\end{figure}

Sin embargo, el objetivo de estas herramientas no fue el motivar los actos ilícitos. 

El uso que le de den la mayoría de usuarios no es el que originalmente se perseguía al crear este tipo de redes, que era el de proteger a aquellas personas que viven en países en los que constantemente se producen abusos contra los ciudadanos de forma sistemática, y el de plantear una solución al problema de censura y represión.

Las \textit{darknets} más conocidas a día de hoy son Tor, I2P y Freenet.

\subsubsection {I2P}

I2P es una darknet o red anónima que nace en el año 2003- Su base funcional es similar a la de Tor. Está escrita en lenguaje Java, con algunos añadidos en C~\cite{article:i2p}.
Una de las ventajas de esta red es el permitir la creación de cualquier tipo de servicio sin mucha complejidad. Es posible poner en marcha servidores HTTP, SSH, FTP o SMB en cualquier instancia I2P sin apenas dificultad. Además, dichos servicios sólo permanecerán accesibles por usuarios internos de I2P.

Existen servicios ocultos de almacenamiento, foros, wikis, documentación, servicios de correo electrónico, repositorios git ocultos, etcétera.

La estructura de una red I2P funciona de la siguiente manera.
Existen una serie de túneles cuyo objetivo es el envío y recepción de paquetes entre los emisores y receptores que se encuentren en la red (ver Figura~\ref{fig:I2PTunneling}). Un túnel es básicamente el conjunto de enrutadores que se encarga de enviar información a un destino determinado, permitiendo una comunicación \textbf{anónima} entre instancias I2P.
En el momento que un usuario lanza una instancia del software I2P automáticamente se convierte en un enrutador y así, en parte de los túneles que crean otras instancias I2P.
Una característica de estos túneles es que son de sentido único. Por este motivo, para que una instancia I2P pueda intervenir en las dos partes de la comunicación (emisión y recepción de paquetes)debecrear túneles de salida y de entrada. Los datos que viajan entre cada enrutador del túnel van cifraados y cada enrutador que recibe el paquete sólo puede acceder a la información correspondiente al siguiente enrutador al que se le debe enviar el paquete.

	\begin{figure}[h]
		\centerline{
			\mbox{\includegraphics[width=5.00in]{images/tunnel_i2p.png}}
		}
		\caption{Funcionamiento básico de la transmisión de paquetes en I2P mediante tunelización.}
		\label{fig:I2PTunneling}
	\end{figure}



\subsubsection {FreeNET}

FreeNET es uno de los proyectos más antiguos que tienen relación con el anonimato en la red. Los inicios del mismo aparecen alrededor del año 2001, y sigue siendo una opción muy viable si bien no es el más utilizado~\cite{article:freenet}. A diferencia de otros, FreeNET se basa en un modelo descentralizado, es decir, que no existen servidores para controlar o gestionar la red y en su lugar cada usuario que se conecta a esta aporta una cantidad de ancho de banda y reserva espacio en su disco duro para almacenar parte de los contenidos que se encuentran en dicha red. 
Este espacio es conocido como \textit{datastore}. Evidentemente no todo usuario tiene acceso completo a estos datos, y sólo el propietario de estos ficheros puede desencriptar sus contenidos utilizando su clave privada.

La principal diferencia con la red Tor, que veremos más adelante, es que el correcto funcionamiento de FreeNET depende de la cantidad de usuarios que se encuentren activos en ese momento, ya que al ser descentralizada, con un gran número de usuarios es muy difícil identificar el origen de una petición determinada. 

Una duda que puede surgir es si los contenidos ubicados en el \textit{datastore} de cada usuario permanecen por siempre. La respuesta es no, ya que al cabo de cierto tiempo, si los datos no son consultados por ningún usuario en la red, terminan siendo eliminados.

Este modelo descentralizado, con el que se almacena en el disco duro información correspondiente de otros usuarios, también tiene sus \textbf{desventajas y peligros}. En el momento en el que una persona suba a la red \textbf{contenido ilegal}, porciones de dicho contenido son almacenadas por el conjunto de usuarios que se encuentren activos en FreeNET. Esto conlleva dos cosas.
La primera es que el autor que ha subido dicho contenido será muy difícil de localizar, debido a que los contenidos en el \textit{datastore} no incluyen metadatos del propietario, ni ninguna información identificativa más allá de la que tengan dichos datos.
El segundo es que un usuario el cual no tenía intenciones maliciosas puede acabar siendo propietario de contenidos que, aunque cifrados, infringen la ley.

Como se ha explicado anteriormente, para acceder a contenidos en FreeNET es necesario conocer la clave que tiene asociado dicho contenido. Existen diferentes tipos de claves:

\begin{itemize}
	\item {Claves SSK}: Este tipo identifica a aquellos contenidos dinámicos que cambian frecuentemente. Las siglas corresponden a \textbf{Signed Subspace Key}. Esta clase de claves funciona con mecanismo de clave pública, es decir que el autor del contenido puede firmarlo y sólo aquellos que sean propietarios de una clave privada pueden realizar modificaciones sobre dicho contenido. Está formada por cinco partes:
	\begin{itemize}
		\item {Hash de clave pública}.
		\item {Clave de descifrado del documento}.
		\item {Configuración criptográfica}.
		\item {Nombre que le ha dado el usuario}.
		\item {Versión de los datos}.
	\end{itemize}
	\item {Claves CHK}: Son las siglas de \textbf{Content Hash Key}, y es la clave más conocida y habitual. Es usada en ficheros estáticos, a diferencia de la clave anterior (como texto, vídeos, documentos PDF, etcétera). Es básicamente el \textit{hash} del contenido del fichero, el cual es unívoco. Está formada por tres secciones:
	\begin{itemize}
		\item {Hash del archivo}.
		\item {Clave de descifrado del archivo}.
		\item {Configuración criptográfica}.
	\end{itemize}
	\item {Claves USK}: Son el tipo de clves más simples. Las siglas vienen del ingés \textbf{Updateable Subspace Keys}. Son realmente las claves que envuelven a las claves SSK. El formato es el siguiente:
	
	{\fontfamily{bch}\selectfont 
		USK@hash,clave	descifrado,config\_criptográfica /sitio\_web, num\_version
	}
	
	\item {Claves KSK}: También conocidas como \textbf{Keyword Subspace Keys}. Se trata de claves que permiten almacenar documentos de texto o páginas etiquetadas en la red de FreeNET.
	El formato es el siguiente: 
	
	{\fontfamily{bch}\selectfont 
		KSK@archivo.tipo
	}
\end{itemize}

\subsubsection {Tor}
Tor es una red anónima que a diferencia de las anteriores mencionadas, está totalmente centralizada. Fue creada en el año 2003 y actualmente es la más usada de entre las tres redes anónimas mencionadas en este subcapítulo. Cuenta con miles de voluntarios en todo el mundo que usan la red y aportan ancho de banda para mejorar la calidad de servicio.

El servicio Tor permite defenderte contra el análisis de tráfico, y avoca por la privacidad y libertad del usuario. La forma en la que lo consigue es haciendo "rebotar" nuestra información en distintos \textbf{nodos Tor} intermedios con el objetivo de que el origen de la información se oculte lo máximo posible.
Conforme los paquetes van avanzando por los nodos intermedios, \textbf{se van cifrando }de forma progresiva, como las capas de una cebolla(de ahí su nombre, \textbf{\textit{The onion router}}).
Dicha encriptación se consigue haciendo uso de las claves públicas de los nodos intermedios. El nodo final es el que descifra el paquete para hacerlo llegar al servidor.
Los usos principales que se le suele dar al software Tor es el de garantizarnos un buen grado de anonimato y el de darnos acceso a los \textbf{\textit{hidden-services} .onion.}.

Sin embargo, Tor es un servicio que puede dar lugar a fugas o \textit{leaks} de información, que pueden facilitar la identificación de un usuario que busca proteger su privacidad. Un ejemplo de ello es que el uso de los nodos Tor \textbf{únicamente soporta el protocolo TCP}. 
Esto conlleva a que, en el momento en el que se utilice un protocolo como UDP o ICMP, las peticiones no pasarán a través de los nodos Tor (también llamado \textbf{circuito virtual}), sino que se establece una conexión directa entre cliente y destino.

Como ha sido dicho, el objetivo de Tor no es únicamente el poder acceder a servicios que se encuentren en Internet, sino que también se utiliza para enrutar las peticiones a servicios que se encuentren alojados en el interior de la red. Estos servicios, en su conjunto, constituyen lo que se conoce como la \textbf{red profunda} de Tor.
Aquí existe un número indeterminado de \textit{hidden services} de todo tipo. Una peculiaridad de estos servicios ocultos es que tienen que estar basados en el protocolo TCP, o de otra manera no pueden estar desplegados sobre la red.
No se utiliza un mecanismo centralizado para la resolución de nombres de dominio, no existen servidores DNS dedicados a la resolución de direcciones IP. En cambio, existe una tabla distribuida de tipo \textit{hash} compuesta por servidores "\textit{HSDir}" que mantienen el registro de los \textit{hidden services} con sus respectivas direcciones.

Crear un servicio oculto con Tor es muy sencillo, y un tutorial del mismo puede ser visualizado en la página web oficial del proyecto Tor:

	{\fontfamily{bch}\selectfont 
		https://www.torproject.org/docs/tor-onion-service.html.en
	}

No obstante, y debido a que este trabajo está orientado al desarrollo de una herramienta \textbf{Python}, a continuación se explicará de manera breve cómo es posible crear un servicio oculto haciendo uso de varias librerías Python, entre ellas \textbf{Stem y Flask}.
La explicación del código fuente puede ser consultada en el \textbf{Anexo A }.

En primer lugar es necesario haber instalado Tor previamente y, si tenemos una instancia de Tor corriendo, detenerla haciendo uso del comando:

	{\fontfamily{bch}\selectfont 
		service tor stop
	}
A continuación, generaremos una contraseña que utilizaremos para iniciar un controlador de la librería Stem (ver Figura~\ref{fig:TorPwd}):

	\begin{figure}[H]
		\centerline{
			\mbox{\includegraphics[width=5.00in]{images/hashed_pwd.png}}
		}
		\caption{\textit{Output} de la contraseña generada.}
		\label{fig:TorPwd}
	\end{figure}
	
Tras haber generado la contraseña \textit{hash}, es necesario reemplazarla por la existente en el campo \textbf{HashedControlPassword} en el archivo:

	{\fontfamily{bch}\selectfont 
		/etc/tor/torrc
	}

Lo único que queda para poder hacer uso del script es aplicar la contraseña propia al parámetro \textit{password} de la función \textit{authenticate} del controlador de Stem(en el caso del ejemplo de arriba, "\textit{trabajofindegrado}"). El resultado sería el siguiente (ver Figura~\ref{fig:controller}):

	\begin{figure}[H]
		\centerline{
			\mbox{\includegraphics[width=5.00in]{images/controller_auth.png}}
		}
		\caption{Configuración del \textit{controller}.}
		\label{fig:controller}
	\end{figure}

Sólo nos queda lanzar el script, pues no es necesario ninguna configuración adicional. El output por consola es el que sigue (ver Figura~\ref{fig:StemOut}):

\begin{figure}[H]
	\centerline{
		\mbox{\includegraphics[width=5.00in]{images/hs_running.png}}
	}
	\caption{\textit{Output} del script cuando es lanzado el servicio oculto.}
	\label{fig:StemOut}
\end{figure}

Tras lanzar el servicio, podemos comprobar que está en funcionamiento entrando en la dirección que nos muestra el anterior output por consola. 

Evidentemente, \textbf{no es necesario estar en la misma red local}, el servicio puede ser visitado sin problema desde cualquier dispositivo, siempre y cuando haga uso del \textbf{proxy Tor} (ver Figura~\ref{fig:hiddenserviceonline}).

\begin{figure}[H]
	\centerline{
		\mbox{\includegraphics[width=5.00in]{images/hs_running2.png}}
	}
	\caption{\textit{Hidden service} en funcionamiento.}
	\label{fig:hiddenserviceonline}
\end{figure} 


Es importante saber que es posible hacer uso de \textbf{\textit{proxychains}}, herramienta de la que ya se ha hablado, \textbf{junto con el servicio Tor}, lo que garantiza un aún mayor grado de anonimato. Esto puede emplearse para escaneos con \textbf{nmap}, establecer conexiones a servidores SSH, etcétera.

Para  ello, basta con ir al archivo de configuración de \textit{proxychains} (Ch. \textbf{2.3.1}) y añadir la dirección IP y el puerto en el cual el servicio Tor está a la escucha. Quedaría de la siguiente forma (ver Figura~\ref{fig:pctor}):
\begin{figure}[H]
	\centering{
		\mbox{\includegraphics[width=4.00in]{images/proxychains_tor.png}}
	}
	\caption{Servicio "tor" junto con otros servidores proxy en la herramienta \textit{proxychains}}
	\label{fig:pctor}
\end{figure} 
Otra herramienta que se beneficia de Tor es \textbf{Onionshare}. 
Onionshare (el cual ha sido integrado en la herramienta Python desarrollada para este proyecto, pero que conviene explicar su funcionamiento) es una herramienta que permite compartir archivos de cualquier tamaño anónimamente. Su funcionamiento es relativamente simple. Convierte a la máquina que ejecuta la herramienta en un servidor web el cual genera una dirección \textbf{.onion} en la que aloja el fichero que elijamos. 
Con ello, cualquiera que conozca el enlace puede \textbf{descargar el fichero a través de Tor}, por ejemplo, con \textbf{Tor Browser}. 
Tiene opciones interesantes como la de desmontar el servidor en el momento en el que alguien descargue el archivo subido (ver Figura~\ref{fig:onionshare}).

\begin{figure}[H]
	\centerline{
		\mbox{\includegraphics[width=5.00in]{images/onionshare1.png}}
	}
	\caption{Ejecución de \textit{Onionshare} por consola}
	\label{fig:onionshare}
\end{figure} 

Una vez lanzado el script (\textbf{también tiene una versión con interfaz, pero para el ejemplo se ha decidido utilizar la versión por consola}) podemos abrir el servicio oculto desde cualquier dispositivo que esté utilizando el servicio Tor para poder descargar el archivo (ver Figura~\ref{fig:onionsharehiddenservice}).

\begin{figure}[H]
	\centerline{
		\mbox{\includegraphics[width=5.00in]{images/onionshare2.png}}
	}
	\caption{Hidden-service creado, con posibilidad de descargar el archivo}
	\label{fig:onionsharehiddenservice}
\end{figure} 

Lo que ocurra después de descargar el archivo depende de la configuración utilizada en el servicio oculto. Puesto que hemos corrido el script con la \textit{flag } --stay-open, el hidden service no será borrado, pudiendo acceder a él otras veces hasta que el servidor pulse Ctrl+C.

\subsection{Otras herramientas de ofuscación}

En apartados anteriores hemos visto varias tecnologías con las que podemos camuflarnos a la hora de acceder a información, ya sea en Internet u otras redes anónimas. Sin embargo, existen incontables tecnologías que poco o nada tienen que ver con las anteriores mencionadas, como las que se expondrán a continuación.

\subsubsection{CacheCloak}

Un usuario que haga uso de un smartphone(es decir, un usuario estándar), debe afrontar riesgos en cuanto a su privacidad en el momento en el que comparte su localización real con los \textbf{LSB} (Servicios Basados en Ubicación). 
Es por esto que nace una tecnología llamada \textit{CacheCloak}, un sistema que permite habilitar una \textbf{anonimización en tiempo real} del servicio de localización.
La manera en la que funciona es simple: Un servidor(de confianza) intermedio recibe la localización real del usuario, la procesa utilizando \textit{CacheCloak} y envía dicha ubicación al LSB.
Además que la calidad del servicio no sufre, el objetivo (conseguir un buen grado de anonimato) se cumple, puesto que los LSB no consiguen identificar al usuario nada más que un corto período de tiempo. 
La forma de conseguirlo es mediante una predicción de movilidad(basada en trayectos ya existentes de otros usuarios) que realiza el servidor(mediante métodos estocásticos) y medidas de entropía.

\subsubsection{AdNauseam}

Las páginas web que visitamos día a día contienen una gran cantidad de \textit{banners} de publicidad, normalmente de diversos temas. Cuando el usuario accede a alguno de ellos, la información sobre los gustos del usuario es recopilada, para así en un futuro promocionar publicidad similar. 

AdNauseam nace como un plug-in de un navegador, y su propósito es el de ofuscar los gustos del usuario haciendo click en cada uno de los \textit{banners} de la página en cuestión. Al entrar en todos los tipos de publicidad diferentes, se consigue que sea más difícil ofrecer una publicidad adaptativa al usuario y, por tanto, se dificulta la identificación del mismo.

\subsubsection{FaceCloak}

FaceCloak es otro plug-in de navegador cuyo objetivo es la ofuscación de los datos personales de un usuario. En este caso, la herramienta limita el acceso que tiene Facebook a los datos personales del usuario. Funciona de la siguiente manera:

\begin{enumerate}
	\item En el momento de registro en Facebook, el plug-in genera datos personales \textbf{falsos}, los cuales serán enviados en los servidores de Facebook.
	\item Una vez que nos registremos, se nos permite agregar nuestros datos personales reales, los cuales daremos acceso a un \textbf{grupo de amigos cerrado}.
	\item Cualquier que pertenezca al anterior grupo, si tiene también activado el plug-in de FaceCloak, al acceder a nuestro perfil podrá ver los datos reales del usuario. En caso contrario sólo verá la información falsa que se envió en el registro a la página.
\end{enumerate}

\subsubsection{Ofuscación de estilometría de documentos}

La estilometría es una métrica que hace usos de elementos de estilo para atribuir un autor a un documento. Puede ser empleada tanto para un documento de texto, ()para ello tendrá en cuenta la longitud de frases, palabras, sintaxis, formato, signos de puntuación entre otros), así como para ficheros de código fuente (asociación de un usuario mediante el estilo del código). 

Existen diversas formas de intentar "evitar" la atribución de autores a documentos. 

Una de ellas, podría consistir simplemente en la \textbf{traducción mecánica} de texto desarrollado a diferentes idiomas y, tras ello, la traducción inversa al idioma original. Esto, si bien es efectivo para textos cortos, hace que en los largos se pierda una cantidad considerable de información y sentido al contenido.

La imitación de otros autores es otra medida que se utiliza comúnmente. Hay ejemplos prácticos de análisis de textos de autores para crear nuevos textos a partir de ellos. Un ejemplo es el de \textit{Insta-Trump} (\url{http://trump.frost.works/}), el cual genera discursos a partir de otros ya existentes del presidente del gobierno de los Estados Unidos. Esto es logrado mediante \textbf{cadenas de Markov}.

Por último, mencionar que existen una serie de ataques de ofuscación contra análsis estilométrico que permiten crear textos sin aparente estilo distintivo, dificultando la labor de identificar al autor.
Un ejemplo es \textbf{Anonymouth}, una herramienta que proporciona al usuario advertencias sobre métricas que debería evitar a la hora de redactar textos(como longitudes de palabra, bigramas, etcétera).

\lsection{Herramientas que ayudan a la identificación de usuarios} \label{sec:identificacion}
\subsection{Introducción}

En apartados anteriores se han visto numerosas herramientas que ayudan a proteger la privacidad de un usuario, así como su grado de anonimato. Todo ello se consigue, como hemos visto, con distintas tecnologías tales como el uso de servidores proxy, VPNs, redes anónimas o ataques de ofuscación en estilometrías. 
Sin embargo, existen también aplicaciones cuyo objetivo es atacar dichas tecnologías. En este apartado se verá como ejemplo \textbf{Tortazo}.

\subsubsection{Tortazo}
Tortazo es, en pocas palabras, un framework de pruebas de auditoría para realizar \textit{pentesting} sobre \textit{\textbf{hidden services}} (independientemente del protocolo que utilicen, SMB, FTP...) y \textbf{repetidores de salida} en Tor.

El proyecto es relativamente reciente y fue creado por \textbf{\textit{Adastra}}, el creador del blog \url{https://thehackerway.net}

Cuenta con tres modos de funcionamiento, y entre estos modos destaca el modo para \textbf{botnet} el cual ataca servicios SSH por fuerza bruta, y permite descubrir sitios .onion (a partir de una dirección .onion parcial o bien a partir de la generación totalmente aleatoria de 16 caracteres en Base32), y un largo etcétera. Cuenta con una arquitectura basada modular que facilita el añadido de \textit{plug-ins} y permite acoplar herramientas de pentesting conocidas como \textbf{nmap o W3AF.}

Vamos a tratar de probar brevemente cada uno de los modos de funcionamiento de esta herramienta:

\begin{itemize}
	\item{\textit{Information gathering}}: El modo de recopilación de información trata de:
	\begin{enumerate}
		\item Establecer una conexión con las autoridades de directorio.
		\item Descargar los descriptores más recientes para así poder conseguir información a partir de los mismos (ver Figura~\ref{fig:tortazo}) (información como  la dirección IP del descriptor, versión de Tor en uso, ancho de banda necesitado para realizar la conexión, etcétera). 
		\item Esta información puede utilizarse para posteriormente realizar un escaneo de puertos mediante \textbf{nmap} a los \textbf{nodos de salida de Tor}, o bien contra \textbf{servicios ocultos}. 
	\end{enumerate}
	
	La herramienta permite recolectar información de los nodos o descriptores conectándose a las autoridades de directorio (\textit{Onion Routers} principales):
	
	{\fontfamily{bch}\selectfont 
		./Tortazo11-linx86\_64 -m linux -a '-sV -A' -n 35 -v
	}
	
	También permite hacerlo conectándose a los espejos/\textit{backups} de las autoridades de directorio:
	
	{\fontfamily{bch}\selectfont 
		./Tortazo11-linx86\_64 -m linux -a '-sV -A' -n 35 -v -d
	}
	
	Por último, cabe la posibilidad de usar los descriptores de una instancia TOR que ya esté en ejecución. Para ello es necesario que una instancia Tor esté en ejecución (por ejemplo, lanzando el \textit{Tor Browser}).
	
	\begin{figure}[H]
		\centerline{
			\mbox{\includegraphics[width=5.00in]{images/tortazo_OR1.png}}
		}
		\caption{Información recogida de los nodos de salida Tor utilizando una instancia en ejecución}
		\label{fig:tortazo}
	\end{figure} 
	
	\item{\textit{Botnet Mode}}: Para hacer uso del modo botnet, es preciso en primer lugar editar el fichero tortazo\_botnet.bot, incluido en las carpetas del proyecto, para incluir las direcciones de servidores que se desea controlar por medio de SSH. El comando para utilizarlo sería 
		{\fontfamily{bch}\selectfont 
			./Tortazo11-linx86\_64 -z all -r 'uname -a;id' -v
		}
	\item{\textit{Onion Repository}}: Aquí la herramienta trata de procesar direcciones .onion a partir de una dirección parcial, o totalmente aleatoria. Esto permite descubrir servicios ocultos en la web profunda de Tor. 
	En el momento que encuentra una dirección .onion funcional, esta es almacenada en una base de datos SQLite.
\end{itemize}
	Por último, mencionar que cuenta con un buen número de plug-ins(\textit{crawler, bruter, w3afplugin...}), cuya información detallada está disponible en la página del proyecto \url{http://tortazo.readthedocs.io/en/latest/}. 
	
\lsection{Debate social}
\label{sec:social}

Existen dos posiciones en cuanto a la protección de la privacidad en aplicaciones de mensajería y en datos personales en redes sociales. Esta división de opiniones se ha visto acrecentada últimamente debido a una serie de sucesos (como actos terroristas organizados mediante herramientas de comunicación con cifrado punto a punto).

Los \textbf{gobiernos}, por una parte, exigen cada vez un mayor control de las empresas tecnológicas para monitorizar los datos compartidos~\cite{article:data} (como propaganda extremista), a la par que menos barreras de seguridad en aplicaciones como \textbf{WhatsApp} o \textbf{Telegram}, para así vigilar comunicaciones y neutralizar futuros atentados. \\
Un ejemplo es la medida que fue tomada en el Departamento de Seguridad Nacional en Estados Unidos (DHS) a partir del 18 de Octubre de 2017, la cual implicaba que toda aquella información tanto de inmigrantes como nacionalizados estadounidenses que se encontrase en redes sociales y diversos foros de Internet pasaría a formar parte del sistema de registro oficial de datos que guarda el gobierno federal~\cite{article:inmigrantes}.

Del mismo modo, existen \textbf{activistas} que consideran que dichas medidas del gobierno son una invasión a la privacidad y no deberían hacerse posible. Matt Cagle, un abogado norteamericano especializado en tecnología, es un ejemplo de ello. Él manifestó en el pasado año que ``las empresas deben ser capaces de proporcionar a sus usuarios una encriptación segura de archivos, lo cual ayude a mantener los datos seguros y protegerlos del gobierno''. También aclaró que ``La privacidad no significa que un individuo esconda algo, es el derecho de poder controlar tu información y el derecho a controlar lo que el mundo y el gobierno sabe de un individuo; es el derecho a tener el espacio para formar tus opiniones, para tener posturas políticas, y para decidir cuándo se quiere llevar esos pensamientos a la esfera pública''~\cite{article:matt}.

El posicionarse en uno de los bandos no depende más que de la opinión personal de cada uno. Sin embargo, una cosa es cierta y es que aún tratando de ocultar tu identidad en Internet lo máximo posible, hay que tener en cuenta que ningún sistema es totalmente fiable y siempre cabe la posibilidad de que la información sea filtrada.

\subsection{Ley De Protección de Datos europea (GDPR)}

La actualización en la Ley de Protección de Datos europea, la cual entrará en vigor en mayo del año 2018, será la encargada de sustituir a la actual Directiva de Protección de datos. 

La medida fue puesta en marcha por la Unión Europea con el objetivo de ofrecer a las personas un mayor control sobre cómo sus datos personales se usan en la red. Con esto, se pretende además mejorar la confianza en la economía digital emergente además de dar a las empresas un entorno jurídico más simplificado, garantizando que la ley de protección de datos sea la misma en todo el mercado único.

La Ley aplicará sobre cualquier empresa que recopile datos personales de ciudadanos europeos. Es importante resaltar que esto será así \textbf{independientemente de si la organización está establecida en la Unión Europea o no}.

Una de las medidas que afectarán positivamente al ciudadano de a pie es que será necesario un \textbf{consentimiento explícito} para la recopilación de datos personales por parte de la empresa, imponiéndose sanciones severas en caso contrario.

Otro hecho importante es que el responsable de tratamiento de datos de los usuarios estará obligado a la notificación de cualquier fallo de seguridad que concierna a los usuarios a la autoridad de protección de datos local~\cite{article:AEPD} (en España es la Agencia Española de Protección de Datos) en un plazo máximo de 72 horas.

Lo cierto es que pese a que a los ciudadanos el GDPR nos favorecerá, \textbf{hay numerosas empresas que podrían caer en chantajes de ciberatacantes} debido al valor que consigue la Personally Identifiable Information (PII)~\cite{article:chantajeGDPR}. Las empresas se podrían ver forzadas a pagarles pues, de lo contrario, un reporte de un ciberatacante sobre una brecha de seguridad por la cual acceder a datos personales podrían llevar a la compañía en cuestión a graves sanciones e indemnizaciones reclamadas de los afectados.

 

