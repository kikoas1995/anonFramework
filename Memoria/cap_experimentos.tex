\chapter{Pruebas}
\label{chap:pruebas}

\subsection{Pruebas}
Durante la etapa de codificación se ha verificado la funcionalidad de cada uno de los módulos implementados mediante distintos tipos de pruebas. 
Se han realizado \textbf{pruebas unitarias} y \textbf{pruebas de integración en el sistema}.

\subsubsection{Pruebas unitarias}

La aplicación, al seguir una estructura modular, facilita la realización de pruebas unitarias.

En el caso de los módulos con funcionalidad específica(como la bandeja de entrada temporal por consola), simplemente se ha comprobado que el módulo cumple su cometido sin errores.

Por otra parte, en el caso de los módulos que contienen varias funciones (el módulo de gestión de la base de datos, ó cualquiera de los módulos en la sección de \textit{accounts}), se ha implementado un método main que ejecute todas y cada una de las funciones del módulo para probar su funcionamiento.

En el caso del módulo de la base de datos, además, se ha comprobado manualmente que la encriptación de la misma es correcta visualizando el contenido de la misma en un editor hexadecimal y comprobando que el contenido de esta desde dicho editor es ilegible.

Por último, cabe mencionar que se ha procurado que el sistema cuente con un buen control de errores en cada uno de los módulos, lo cual impida cierres forzosos.

\subsubsection{Pruebas de integración}

Para las pruebas de integración, se ha comprobado que es posible lanzar cualquiera de los módulos Python desde los módulos \textbf{main}, tanto el interactivo con el usuario como el \textit{script} de ejecución automática, sin que ello ocasione ningún fallo del sistema.

Para ello se ha comprobado el arranque de \textbf{cada uno} de los módulos desde los métodos \textit{main} del programa principal.

Asimismo, puesto que un objetivo de la aplicación era preservar la privacidad del usuario, se ha comprobado que usando servicios Tor proporcionados por la aplicación, efectivamente camuflamos nuestra dirección de origen. También se ha comprobado los cambios en las cabeceras del navegador para el módulo de \textit{mailer}.

Por último, se ha probado a lanzar la aplicación a través de la herramienta \textit{proxychains} y también se ha probado haciendo uso de una VPN a través de \textit{OpenVPN}, ambas con resultados satisfactorios.

