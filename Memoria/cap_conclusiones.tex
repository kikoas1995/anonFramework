\chapter{Conclusiones y trabajo futuro}
\label{chap:conclusiones}
\section{Conclusiones}
El estudio sobre diversas tecnologías que me ha proporcionado este
proyecto ha motivado aún más mi interés por la seguridad informática,
así como la concienciación sobre la importancia del derecho a la
privacidad.  Es un hecho que en el momento en que nos conectamos a
Internet estamos siendo observados tanto por \textbf{entidades
  privadas} como por aquellas otras financiadas por los gobiernos, las
cuales monitorizan y registran cada una de nuestras actividades en la
red. Este hecho es especialmente destacable en países como
\textbf{Estados Unidos}, donde mediante las medidas de vigilancia y
vigencia de leyes mordaza se le permite al gobierno el forzar
secretamente a compañías a garantizar el acceso a datos de clientes y
usuarios, transformando con ello un servicio en una herramienta de
vigilancia.

La importancia que le dan los usuarios a la privacidad se ha ido
viendo incrementada en los últimos años debido a numerosos hechos (como
filtraciones de datos en páginas web reconocidas), pero hay otra gran
masa que piensa que no es un problema que sus datos personales
circulen por la red.  En realidad la exigencia de un mayor grado de
privacidad no sería tan necesario de no ser por los continuos abusos y
el control que tienen las grandes empresas con todos los usuarios que
utilizamos diariamente Internet.

Debido a estas surgentes invasiones a la privacidad, es el momento de
considerar el uso de \textbf{tecnologías que apoyen el derecho a la
  privacidad}.

Como se ha visto en el capítulo 2.3, el uso de servidores proxy, VPNs
o redes anónimas alternativas a Internet son muy buenas opciones para
ello. En la herramienta desarrollada en este proyecto se ha querido
partir de la base de algunas de estas tecnologías, como
\textit{mailers }anónimos y el servicio Tor, para proporcionar al
usuario final una herramienta que permita realizar tareas del día a
día en Internet de una forma más anónima.  La aplicación diseñada
puede considerarse una primera aproximación para proteger nuestros
datos personales, y podría utilizarse con medidas adicionales como las
ya mencionadas en este documento para incrementar aún más el grado de
privacidad.

Es importante destacar que la metodología empleada es de utilidad con 
respecto a dos normativas cuya puesta en vigor está a la vuelta de la 
esquina, como son la ya citada GDPR y \textbf{PSD-2}~\cite{article:psd2}.
Mientras que \textit{Payment Service Directive II}, centrada en el ámbito 
financiero, obliga a las instituciones financieras a permitir el acceso de 
terceros (mediante \textit{screen-scraping} o mediante APIs) a los datos de los usuarios, también se debe tener en cuenta la normativa de la GPDR, y lo que esta 
supone con respecto a la protección del dato privado.

Para terminar, conviene mencionar que el estudio necesario para el
desarrollo de esta herramienta ha reforzado mis conocimientos tanto de
bibliotecas que desconocía como del manejo con el propio lenguaje
Python.

\section{Trabajo futuro}

Pese a que se ha realizado un estudio de muchas tecnologías contra la protección de la privacidad, gran parte de ellas no han sido integradas en la aplicación final debido a la envergadura del proyecto.

\begin{itemize}
	\item \textbf{Actualización continua de los módulos del paquete \textit{accounts}}: La utilización de herramientas de \textit{web-scraping} para la examinación de sitios web con el fin de registrar usuarios automáticamente tiene un problema, y es que \textbf{depende totalmente de la estructura del sitio web.} Es decir, que en el momento en el que la web efectúe algún cambio con respecto a las páginas de formulario de registro o inicio de sesión, será necesario actualizar el módulo correspondiente a dicha página para hacerla acorde a la nueva estructura. Por ello, el mantenimiento constante de la aplicación es algo esencial para su correcto funcionamiento.
	\item \textbf{Implementación de nuevos módulos en el paquete de \textit{accounts}}: Se han desarrollado varios módulos de ejemplo en páginas diferentes entre sí. No obstante, la ampliación del número de páginas disponibles no supondría un gran esfuerzo, debido a la posibilidad de reutilizar gran parte del código de otros módulos. También se podría considerar el hecho de añadir más acciones automáticas en cada página.
	\item \textbf{Adición de nuevas tecnologías existentes}: Habilitar un módulo que conecte automáticamente a un servidor \textit{proxy} o una VPN, así como la inclusión de nuevas medidas de ofuscación de estilometrías.
	\item \textbf{Gestor completo de la base de datos}: Una mejora a tener en cuenta sería la de incluir un gestor que permita insertar usuarios manualmente, borrarlos, visualizar los usuarios en cada tabla, todo ello directamente por consola.
	\item \textbf{Gestor completo de la base de datos}: Una mejora a tener en cuenta sería la de incluir un gestor que permita insertar usuarios manualmente, borrarlos, visualizar los usuarios en cada tabla, etcétera. Todo ello directamente por consola.	
	\item \textbf{Interfaz gráfica}: Podría estudiarse la posibilidad incluir un entorno gráfico vistoso que permita realizar cada una de las funcionalidades que ofrece la herramienta.
	\item \textbf{Base de datos compartida}: La compartición de las tablas de usuarios registrados de la base de datos entre dos o más usuarios de la herramienta haría innecesario el tener que registrar un gran número de usuarios individualmente, al mismo tiempo que favorecería a la ofuscación de emisor, al ser utilizado en más de un equipo.
	\item \textbf{Sistema de \textit{feedback}}: Habilitar un sistema de opiniones o recomendaciones al desarrollador por parte de los usuarios de la aplicación.	
	\item \textbf{Mejoras de tiempo de ejecución}: Se podría conseguir mediante el uso de \textit{WebDrivers} más ligeros como PhantomJS o la disminución del retardo a la hora de rellenar formularios de registro.	
	\item \textbf{Sistema de identificación de usuarios}: En versiones futuras se tendrá en cuenta el poder incluir múltiples factores de autenticación (\textit{MFA}) a la hora de acceder a las distintas funcionalidades del sistema.	
        \item \textbf{Pruebas de usabilidad}: En versiones futuras se tomará nota en cuanto a la posibilidad de integrar pruebas de usabilidad.	

\end{itemize}


Todas estas mejoras podrían realizarse mediante ayudas de desarrolladores debido a que la aplicación es de código libre y abierto (el enlace al repositorio está incluído en el Anexo~\ref{Anexo:repositorio}).

\newpage \thispagestyle{empty} % Página vacía 