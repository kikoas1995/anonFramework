\chapter*{Resumen}

\section*{Resumen}
La tecnología ha tenido un gran impacto social en los últimos años, y cada vez va formando más parte de nuestras vidas.\\
Las grandes empresas de software lo saben, y muchas de ellas tienen
objetivos de dudosa moralidad, como la recopilación de información
personal para poder ofrecernos unos servicios más apropiados para cada
usuario. Entre esta información podemos encontrar alguna tan íntima
como la dirección, el número de teléfono o incluso los lugares que
frecuentamos, y basta con un inicio de sesión en
una red social para que dichos datos fluyan por la red.  Esto mismo ha
llevado a muchos el cuestionarse hasta qué punto nuestro derecho a la
privacidad se está viendo comprometido y, pese a que mucha otra gente
no de la importancia que merece a este tema, lo cierto es que no son
pocas las herramientas que han aparecido para ayudarnos a que podamos
navegar por internet de una forma \textit{menos pública}.  De ahí nace
la motivación de este proyecto, de proporcionar varias formas de
realizar tareas en Internet de la forma más anónima posible, a partir
de una única herramienta. Esta herramienta ha sido codificada en
\textit{Python}, y cuenta con varios paquetes de módulos, cada uno con
una funcionalidad distinta.

El primer paquete cuenta con varias funcionalidades relacionadas con
el envío y recepción de correos electrónicos. Un módulo de ese paquete
nos permite hacer uso de un \textit{remailer} para proteger lo máximo
posible la identidad del emisor de dicho correo, otro módulo que
permite enviar correos pero esta vez permitiendo al usuario
\textit{ofuscar} el contenido del mensaje y por último una bandeja de
entrada temporal que permite ver en tiempo real los correos que envían
a dicha cuenta.

El segundo paquete está relacionada con la automatización de procesos
de registro, inicio de sesión e incluso ciertas acciones en páginas
web. Para ello se ha hecho uso de \textit{web-scraping} en redes
sociales, diarios y foros de discusión. Permite, de esta forma,
utilizar cuentas de correo volátiles con el fin de \textit{anonimizar}
la identidad al ingresar dichas páginas. Además, se deja a disposición
del usuario el poder utilizar el \textit{servicio Tor} para así
mejorar aún más la privacidad. Las cuentas de usuario generadas en el
registro de las páginas son almacenadas en una base de datos local
\textit{encriptada}.

El último paquete está relacionado con la posibilidad de
\textit{enmascarar tráfico web}, uso de \textit{VPNs}, de
\textit{proxies}, y más funcionalidades que están descritas en este
documento.

Cabe destacar que el presente proyecto usa una metodología
\textit{open source} y permite fácilmente la inclusión de módulos
adicionales para aumentar aún más su funcionalidad.

\section*{Palabras Clave}
\textit{remailer, ofuscar, web-scraping, anonimizar, servicio Tor, encriptada, enmascarar tráfico web, VPN, proxies, open source}.
\newpage

%-------------------------------------------------------------------------------------------------------------------------------------
\section*{Abstract}
Technology has had a major social impact in the last decades, and has become part of our daily life.\\
The most influential software companies are conscious about this, and they have ethically questionable objectives, like a complete collection of personal information in order to offer us better and customized services. Amongst this information there is sensible data, such as our personal address, our telephone number or even the places we usually visit~\cite{article:google}. Logging in a page is the only thing we need to do in order to let our private information spread on the internet.
%Quizá, y lo que es un problema mayor, es que una gran parte de nosotros no somos conscientes de lo que realmente esto supone, de que dichas corporaciones saben nuestra actividad, estado de ánimo y prácticamente todo de nuestro día a día.\\

This condition has led many people think to what extent is our privacy compromised. Many others do not concern this topic but, even so, there has been a recent increase in the number of tools which let us navigate on the web in a less-public way. This project is born from this idea, the idea of providing many simple and daily functionalities in the most anonymous way possible. It has been coded in Python and it is composed by several source packages, eac h one with different purposes.\\
The first package of the tool has three main functionalities, all of them related to the seng and reception of e-mails. The first functionality lets us make use of a \textit{remailer} which will protect the identity of the sender. Another module provides us a service of sending e-mails, with an option of \textit{obfuscating} the content of the message and. Lastly, a real-time volatile inbox, which lets us see from console each e-mail that arrives to our temporary account.\\
The second package is related with the automation of signing up, logging in and other functionalities in web pages, such as newspapers, social networks, and discussion forums. In order to deal with these tasks it has been required to make use of \textit{web-scraping}. This package basically lets us to sign in a page using a volatile e-mail, without the neccessity of using our personal one with the purpose of \textit{anonimyzing} our identities. It also has an option of using the \textit{Tor service} in these proccesses to hide our identity even more. Accounts generated in this package is stored in an \textit{encrypted} database. \\
Last package is related with the possibility of \textit{masking web traffic}, using \textit{VPNs}, \textit{proxies}, and more functionalities which will be described in this document.\\
Finally, there is a noteworthy effort of developing this project with an \textit{open source} metodology, letting the final user include more modules in it easily.\\

\section*{Key words}
\textit{remailer, obfuscate, web-scraping, anonimyzing, Tor service, encrypted, masking web traffic, VPN, proxies, open source}.

